\documentclass[letter,14pt]{extarticle}
\usepackage[utf8]{inputenc}
\usepackage{fontspec}
\usepackage{xcolor}
\usepackage{draftwatermark}
\usepackage[margin=0.5in]{geometry}
\usepackage{tikz}
\usetikzlibrary{calc}
\usepackage{eso-pic}
\usepackage{tabularx}
\usepackage{graphicx}
\usepackage{hyperref}
\usepackage[T1]{fontenc}
\AddToShipoutPictureBG{%
\begin{tikzpicture}[overlay,remember picture]
\color{amarillo}
\draw[line width=14pt]
    ($ (current page.north west) + (0.05cm,-0.1cm) $)
    rectangle
    ($ (current page.south east) + (-0.15cm,0.15cm) $);
\end{tikzpicture}
}

\setmainfont[Ligatures=TeX]{Montserrat-Light.ttf}
\newfontfamily\normalfont[Ligatures=TeX]{Montserrat-Light.ttf}
\newfontfamily\italicafont[Ligatures=TeX]{Montserrat-LightItalic.ttf}
\newfontfamily\mediumfont[Ligatures=TeX]{Montserrat-Medium.ttf}
\newfontfamily\negritafont[Ligatures=TeX]{Montserrat-Regular.ttf}
\newfontfamily\negritaitalicafont[Ligatures=TeX]{Montserrat-Italic.ttf}
\newfontfamily\longreachfont[Ligatures=TeX]{Longreach.otf}

\DeclareTextFontCommand{\normal}{\normalfont}
\DeclareTextFontCommand{\italica}{\italicafont}
\DeclareTextFontCommand{\medium}{\mediumfont}
\DeclareTextFontCommand{\negrita}{\negritafont}
\DeclareTextFontCommand{\negritaitalica}{\negritaitalicafont}
\DeclareTextFontCommand{\longreach}{\longreachfont}

\renewcommand{\thefootnote}{\fnsymbol{footnote}}

\usepackage{etoolbox}

\newcommand{\footnotetextnumbering}{\alph{footnote}}
\makeatletter
\patchcmd{\footnote}% <cmd>
  {\@footnotemark}% <search>
  {\protected@xdef\@thefntextmark{\footnotetextnumbering}%
    \@footnotemark}% <replace>
  {}{}% <success><failure>
\def\@makefntextmark{\hbox{\@textsuperscript{\normalfont\@thefntextmark}}}
\patchcmd{\@makefntext}{\@makefnmark}{\@makefntextmark}{}{}
\makeatother


\definecolor{amarillo}{RGB}{245,184,12}
\pagenumbering{gobble}
\setlength{\arrayrulewidth}{4pt}

\SetWatermarkText{\includegraphics[angle=-45]{Logo.png}}
\SetWatermarkScale{0.7}
\SetWatermarkLightness{1}

\usepackage[spanish]{babel}
\usepackage[spanish]{translator}    

\deftranslation[to=spanish]{January}{Enero}
\deftranslation[to=spanish]{February}{Febrero}
\deftranslation[to=spanish]{March}{Marzo}
\deftranslation[to=spanish]{April}{Abril}
\deftranslation[to=spanish]{May}{Mayo}
\deftranslation[to=spanish]{June}{Junio}
\deftranslation[to=spanish]{July}{Julio}
\deftranslation[to=spanish]{August}{Agosto}
\deftranslation[to=spanish]{September}{Septiembre}
\deftranslation[to=spanish]{October}{Octubre}
\deftranslation[to=spanish]{November}{Noviembre}
\deftranslation[to=spanish]{December}{Diciembre}
\deftranslation[to=spanish]{Mon}{Lun}
\deftranslation[to=spanish]{Tue}{Mar}
\deftranslation[to=spanish]{Wed}{Mié}
\deftranslation[to=spanish]{Thu}{Jue}
\deftranslation[to=spanish]{Fri}{Vie}
\deftranslation[to=spanish]{Sat}{Sab}
\deftranslation[to=spanish]{Sun}{Dom}

\usetikzlibrary{calc}
\usetikzlibrary{calendar}
% \renewcommand*\familydefault{\sfdefault}

% User defined

\begin{document}

\pagecolor{white}

\begin{center}
    \fontsize{50}{60}\selectfont {\color{amarillo}\longreach{Pauta guía de comunidad}}
\end{center}

\hfill \\ \hfill
{\medium{Estructura}}
{\color{amarillo} \vspace{4pt} \hrule height 4pt }
\hfill \\ \hfill
\normal{Todas las oraciones y motivaciones están en letra \italica{cursivas}. Estas son leídas en voz alta, el resto son instrucciones para el guía de cada grupo.

Antes de partir la celebración el dueño de casa puede preparar el lugar armando un pequeño altar. Este consta de tres signos: \negrita{mantel}, signo con el que destacamos la mesa que vamos a usar como altar, convirtiendo este espacio en un lugar sagrado; \negrita{cruz}, signo de vida, del misterio de la fe cristiana: por la muerte se encuentra la vida... Jesús murió, pero resucitó, esta vivo entre nosotros y nos habla; y \negrita{cirio}, signo de la presencia entre nosotros de Cristo Resucitado por medio del Espíritu Santo, es la luz de Cristo que nos ilumina.}

\hfill \\ \hfill
{\medium{Oración Inicial}}
{\color{amarillo} \vspace{4pt} \hrule height 4pt }
\hfill \\ \hfill
\normal{Nos persignamos e invocamos al Espíritu Santo para que nos ilumine y guíe durante la reflexión que vamos a tener.

\italica{“Ven Espíritu Santo, llena los corazones de tus fieles y abrázalos en el fuego de tu amor. Envía Señor tu Espíritu y todas las cosas serán creadas.
R: Y renovarás la faz de la tierra.”}}

\hfill \\ \hfill
{\medium{Motivación al Evangelio}}
{\color{amarillo} \vspace{4pt} \hrule height 4pt }
\hfill \\ \hfill
\normal{Es recomendable que el guía de cada grupo prepare su propia motivación a escuchar el Evangelio, pero en caso de no hacerlo puede leer una preparada. Si decide prepararla, es importante recalcar que la motivación no debe ser demasiado larga, ni muy explicativa. La idea es que una vez hecha la motivación los que participan de la liturgia quieran escuchar la lectura. Es como si estuviesen invitándolos a comer un plato muy rico, no les describirían los ingredientes del plato, sino que darían razones para comerlo.

\hfill \\ \hfill
'¡Motivación!'
\hfill \\ \hfill
    
Se puede dejar un rato de silencio antes de leer el Evangelio para entrar con la mente limpia y sin preocupaciones.
    
\newpage}

\newpage
{\noindent \medium{Evangelio}}
{\color{amarillo} \vspace{4pt} \hrule height 4pt }
\hfill \\ \hfill
\normal{Puede imprimirse el Evangelio (si no tienen Biblias) y cada uno seguir la lectura desde su papel. \href{¡link!}{\color{amarillo}Evangelio para imprimir}
\hfill \\ \hfill
    
\italica{Lectura del santo evangelio según san X (a, b-c):}
\hfill \\ \hfill
    
\negritaitalica{"
\textsuperscript{a}Bla bla 
\textsuperscript{b}bla bla.
"}

\hfill \\ \hfill
    
\italica{Palabra del Señor. R: Gloria a ti Señor Jesús}}

\hfill \\ \hfill
{\medium{Actividades}}
{\color{amarillo} \vspace{4pt} \hrule height 4pt }
\hfill \\ \hfill
\normal{Meditar es masticar, rumiar, repetir, reflexionar, recordar. Es dentro de ese ejercicio que la meditación busca en primer lugar reconocer a la persona de Jesucristo en la lectura para luego descubrir lo que Él me quiera decir a mí, aquí y ahora, a través de ella. La pregunta que busca responder la meditación es “¿qué me dice el Señor?”. Para esto cada guía de comunidad puede proponer su propia actividad, sin embargo algunas sugeridas son:

\begin{enumerate}
  \item 
\end{enumerate}
    
Si quieren pueden mezclar estas actividades, elegir algunas, o inventar otra nueva. Este espacio está muy abierto a la creatividad del guía.  
Terminamos rezando un Padre Nuestro y poniendo en común algunas intenciones.}

\hfill \\ \hfill
{\medium{Propósito Comunitario}}
{\color{amarillo} \vspace{4pt} \hrule height 4pt }
\hfill \\ \hfill
\normal{\input{semana_prueba2/6_proposito.tex}}

\hfill \\ \hfill
{\medium{Oración Final}}
{\color{amarillo} \vspace{4pt} \hrule height 4pt }
\hfill \\ \hfill
\normal{\italica{“Querido Jesús, te agradecemos por esta reflexión. Te pedimos que a través de tu ejemplo en el Evangelio, nos enseñes a amar como Tú lo hiciste y a ser jóvenes que viven Tu palabra. Que nuestra vida sea un constante caminar en la fe para que podamos conocer a Dios y que, como San Cristóbal, podamos ser portadores tuyos con nuestros amigos y en nuestras familias. Amén”}}

\newpage
\begin{center}
    \fontsize{50}{60}\selectfont {\color{amarillo}\longreach{Anexos}}
\end{center}
% \hfill \\ \hfill
\normal{\input{semana_prueba2/8_anexos.tex}}

\def\iyear{2020} % Año de inicio
\def\fyear{2020} % Año de fin
\def\imonth{11} % Mes de inicio
\def\iday{3} % Día de inicio
\def\fmonth{11} % Mes de fin
\def\fday{10} % Día de fin

\def\lunes{\\} % Lo que va el día lunes
\def\martes{\\} % Lo que va el día martes
\def\miercoles{\\} % Lo que va el día miércoles
\def\jueves{\\} % Lo que va el día jueves
\def\viernes{\\} % Lo que va el día viernes
\def\sabado{\\} % Lo que va el día sábado
\def\domingo{\\} % Lo que va el día domingo

\hfill \\ hfill
{\medium{Esta semana}}
{\color{amarillo} \vspace{4pt} \hrule height 4pt }
\hfill \\ \hfill
\normal{\negrita{Lunes:}

\negrita{Martes:}

\negrita{Miercoles:}

\negrita{Jueves:}

\negrita{Viernes:}

\negrita{Sabado:}

\negrita{Domingo:}}

\begin{center}
\hspace*{-0.5cm}
\begin{tikzpicture}[every day/.style={anchor = north}, font=\fontsize{12}{15}\selectfont]
\calendar[
  dates=\iyear-\imonth-\iday to \fyear-\fmonth-\fday,
%   dates=\iyear-10-07 to \fyear-10-13,
  name=cal,
  day yshift = 3em,
  day code=
  {
    \node[name=\pgfcalendarsuggestedname,every day,shape=rectangle,
    minimum height= 2.3cm, text width = 2.3cm, draw = gray, text depth = 2.4 cm]{
    \ifdate{Monday}{\normal{lunes}}{}\ifdate{Tuesday}{\normal{\martes}}{}\ifdate{Wednesday}{\normal{\miercoles}}{}\ifdate{Thursday}{\normal{\jueves}}{}\ifdate{Friday}{\normal{\viernes}}{}\ifdate{Saturday}{\normal{\sabado}}{}\ifdate{Sunday}{\normal{\domingo}}{}};
    \draw (-1.8cm, -.1ex) node[anchor = west]{};
  },
  execute before day scope=
  {
    \ifdate{workday}
    {
      % Shift right
      \tikzset{every day/.style={fill=white}}
      \pgftransformxshift{2.6cm}
      % Print month name 
      \draw (0,0)node [shape=rectangle, minimum height= .53cm,
        text width = 2.3cm, fill = amarillo, text= black, draw = amarillo, text centered]
        {\fontsize{12}{15}\selectfont\negritafont\pgfcalendarweekdayshortname{\pgfcalendarcurrentweekday} \tikzdaytext};
    }{}
    \ifdate{Saturday}{
      % Shift right
      \pgftransformxshift{2.6cm}
      % Print month name 
      \draw (0,0)node [shape=rectangle, minimum height= .53cm,
        text width = 2.3cm, fill = amarillo, text= black, draw = amarillo, text centered]
        {\fontsize{12}{15}\selectfont\negritafont\pgfcalendarweekdayshortname{\pgfcalendarcurrentweekday} \tikzdaytext};
    \tikzset{every day/.style={fill=amarillo!10}}}{}
    \ifdate{Sunday}{
      % Shift right
      \pgftransformxshift{2.6cm}
      % Print month name 
      \draw (0,0)node [shape=rectangle, minimum height= .53cm,
        text width = 2.3cm, fill = amarillo, text= black, draw = amarillo, text centered]
        {\fontsize{12}{15}\selectfont\negritafont\pgfcalendarweekdayshortname{\pgfcalendarcurrentweekday} \tikzdaytext};
    \tikzset{every day/.style={fill=amarillo!20}}}{}
  },
  execute at begin day scope=
  {
    % each day is shifted down according to the day of month
    \pgftransformyshift{-1.8 cm}
  }
];
\end{tikzpicture}

\end{center}

\end{document}